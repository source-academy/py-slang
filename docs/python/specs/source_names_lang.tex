\subsection*{Names}

Names\footnote{
In
\href{https://docs.python.org/3/reference/lexical_analysis.html}{
\color{DarkBlue} Python 3.13 Documentation},
these names are called \emph{identifiers}.
} start with \verb@_@, or a
letter\footnote{
By \emph{letter}
we mean \href{http://unicode.org/reports/tr44/}{\color{DarkBlue}Unicode} letters (L) or letter numbers (NI).
} and contain only \verb@_@, 
letters or digits\footnote{
By \emph{digit} we mean characters in the
\href{http://unicode.org/reports/tr44/}{Unicode} categories
Nd (including the decimal digits 0, 1, 2, 3, 4, 5, 6, 7, 8, 9), Mn, Mc  and Pc. 
}. Reserved words\footnote{
By \emph{reserved word} we mean any of:
$\textbf{\texttt{False}}$, $\textbf{\texttt{await}}$, $\textbf{\texttt{else}}$, $\textbf{\texttt{import}}$, $\textbf{\texttt{pass}}$, $\textbf{\texttt{None}}$, $\textbf{\texttt{break}}$, $\textbf{\texttt{except}}$, $\textbf{\texttt{in}}$, $\textbf{\texttt{raise}}$, $\textbf{\texttt{True}}$, $\textbf{\texttt{class}}$, $\textbf{\texttt{finally}}$, $\textbf{\texttt{is}}$, $\textbf{\texttt{return}}$, $\textbf{\texttt{and}}$, $\textbf{\texttt{continue}}$, $\textbf{\texttt{for}}$, $\textbf{\texttt{lambda}}$, $\textbf{\texttt{try}}$, $\textbf{\texttt{as}}$, $\textbf{\texttt{def}}$, $\textbf{\texttt{from}}$, $\textbf{\texttt{nonlocal}}$, $\textbf{\texttt{while}}$, $\textbf{\texttt{assert}}$, $\textbf{\texttt{del}}$, $\textbf{\texttt{global}}$, $\textbf{\texttt{not}}$, $\textbf{\texttt{with}}$, $\textbf{\texttt{async}}$, $\textbf{\texttt{elif}}$, $\textbf{\texttt{if}}$, $\textbf{\texttt{or}}$, $\textbf{\texttt{yield}}$.
These are all reserved words, or keywords of the language that cannot be used as ordinary identifiers.
} are not allowed as names.

Valid names are \verb@x@, \verb@_45@, and $\mathtt{\pi}$,
but always keep in mind that programming is communicating and that the familiarity of the
audience with the characters used in names is an important aspect of program readability.
