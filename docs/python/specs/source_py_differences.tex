\section*{Deviations from native Python}

We intend the Python \S $x$ to be a conservative extension of 
native Python: Every correct Python \S $x$ program should behave 
\emph{exactly} the same using a Python \S $x$ implementation, as it does 
using a native Python implementation. We assume, of course, that 
suitable libraries are used by the TypeScript implementation, to 
account for the predefined names of Python \S $x$. 
This section lists some exceptions where we think a Python \S $x$ 
implementation should be allowed to deviate from the native Python 
specification, for the sake of internal consistency and esthetics.

\begin{description}
\item[{Output Behavior Differences Between py-slang and Standard Python REPL:}]
  Python provides an interactive mode, where users can type and immediately 
  evaluate Python expressions line by line. This mode is entered when you 
  simply type python in a terminal without specifying a file. This is commonly 
  referred to as the Python REPL. 
  In the standard Python REPL, any evaluated expression automatically has its 
  result printed to the console, even if the user does not explicitly call 
  \texttt{print()}. For example:

  \begin{lstlisting}
>>> 1 + 2
3
>>> "hello"
'hello'
  \end{lstlisting}

  This is because Python's REPL implicitly displays the return value of each 
  expression, unless it is None.

  In contrast, py-slang adopts a more controlled and minimalistic REPL behavior:
  Only expressions explicitly passed to \texttt{print()} produce output. If the 
  expression is evaluated without a print call, no output will appear, even 
  though a value is computed.

  For example, in py-slang:

  \begin{lstlisting}
print(1 + 2)  	# Outputs: 3
  \end{lstlisting}

  This design aligns with the pedagogical goals of Source Academy, reinforcing 
  the idea that output should be intentional and explicit, helping students 
  better understand the role of side effects and output operations.
\end{description}